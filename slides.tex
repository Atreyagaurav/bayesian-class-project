\documentclass{beamer}
\usepackage[utf8]{inputenc}

\usepackage{graphicx}

\graphicspath{{images/}}

\usepackage{amsmath}
\usepackage{amssymb}
\usepackage{booktabs}
\usepackage{adjustbox}
\usepackage{appendixnumberbeamer}
\usetheme{Berlin}
\useinnertheme{circles}

%% To remove subsection bar on header:
%% From: https://tex.stackexchange.com/questions/257394/modifying-headline-in-beamer-berlin
\setbeamertemplate{headline}
{%
  \begin{beamercolorbox}[colsep=1.5pt]{upper separation line head}
  \end{beamercolorbox}
  \begin{beamercolorbox}{section in head/foot}
    \vskip2pt\insertnavigation{\paperwidth}\vskip2pt
  \end{beamercolorbox}%
  \begin{beamercolorbox}[colsep=1.5pt]{lower separation line head}
  \end{beamercolorbox}
}

\title{Missing Variable Inference using Bayesian Approach on a River Network}
\author{Gaurav Atreya}
\institute{
 University of Cincinnati}
\date{2025 April 25}
\logo{\includegraphics[width=.05\linewidth]{images/uc-logo.png}}

\usepackage{tikz}
\usetikzlibrary{tikzmark}

\newcommand{\Node}[3][0]{%
  \tikz[overlay,remember picture]{\draw (#1 + 0.5, 0.1) circle [radius=0.2] node (#2) {#3};%
  }}


\usepackage[langlinenos]{minted}

\definecolor{codebg}{rgb}{0.90,1,0.90}
\setminted[jags]{
  bgcolor=codebg,
  breaklines=true,
  linenos,
  fontsize=\scriptsize,
  frame         = single                   ,
  framesep      = 1mm                      ,
  label         = \textrm{\textbf{JAGS Model}},
  labelposition = topline}

\setminted[r]{
  bgcolor=codebg,
  breaklines=true,
  linenos,
  fontsize=\scriptsize,
  frame         = single                   ,
  framesep      = 1mm                      ,
  label         = \textrm{\textbf{R Code}},
  labelposition = topline}


\begin{document}

\begin{frame}
  \titlepage
\end{frame}

\section{Introduction}

\begin{frame}{River Network}
  \begin{columns}
    \column{0.4 \textwidth}
    A River Network in consideration have to be a directed tree:
    \begin{itemize}
    \item Directed,
    \item Acyclic,
    \item Planar, and
    \item Weakly Connected.
    \end{itemize}

    \column{0.5 \textwidth}
    \includegraphics[width=.9\textwidth]{./images/network.pdf}

  \end{columns}

  \vspace{0.5cm}
    
    And the variable of interest will be the data at the nodes of the graph.

    \vfill
    \pause
    
    {\color{blue}Key Terms}: Node, Output, Inputs, Leaf, and Outlet.
    
\end{frame}

\begin{frame}{Data}

  We have, basin area as independent variable, and annual average river flows as dependent variables.

  \vspace{0.5cm}

  \begin{columns}
    \column{0.5\textwidth}
\begin{adjustbox}{width=0.9\textwidth, totalheight=\textheight-2\baselineskip-2\baselineskip,keepaspectratio}
  \begin{tabular}{lrrcrr}
    \toprule
    Nodes Connection & Area (\(x\)) & 2005 (\(y_{1}\)) & ... & 2023 (\(y_{19}\)) & 2024 (\(y_{20}\))\\[2mm]
    \midrule
\Node[0]{0}{0}   & 5879.03 & 2180.76 & \(\hdots\) & 2124.50 & 2386.84 \\[2mm]
\Node[1]{1}{1}   & 1424.00 &  784.83 & \(\hdots\) &  479.92 & 498.48 \\[2mm]
\Node[1]{2}{2}   &  325.50 &  155.46 & \(\hdots\) &  114.27 & 119.66 \\[2mm]
\Node[1]{3}{3}   &  169.97 &   90.01 & \(\hdots\) &   66.32 & 72.78 \\[2mm]
\Node[2]{4}{4}   &  491.00 &  302.54 & \(\hdots\) &  188.94 & 252.79 \\[2mm]
\Node[2]{5}{5}   &  262.85 &   81.56 & \(\hdots\) &  117.81 & 117.91 \\[2mm]
\Node[0]{6}{6}   & 4205.12 & 2497.53 & \(\hdots\) & 1488.69 & 1603.67 \\[2mm]
\Node[1]{7}{7}   &  463.92 &  315.63 & \(\hdots\) &  152.38 & 180.38 \\[2mm]
\Node[1]{8}{8}   & 2549.06 & 1535.94 & \(\hdots\) &  835.36 & 1013.28 \\[2mm]
\Node[1]{9}{9}   & 1469.51 &  843.90 & \(\hdots\) &  420.13 & 544.08 \\[2mm]
\Node[0]{10}{10} & 1278.31 &  743.12 & \(\hdots\) &  507.92 & 623.50 \\[2mm]
\Node[0]{11}{11} & 1019.56 &  528.46 & \(\hdots\) &  385.48 & 420.40 \\[2mm]
\Node[0]{12}{12} &   98.27 &   60.85 & \(\hdots\) &   41.15 & 36.50 \\[2mm]
\bottomrule
  \end{tabular}
  
\tikz[overlay, remember picture]{
  \path[->] (1) edge (0);
  \path[->] (2) edge (1);
  \path[->] (3) edge (2);
  \path[->] (4) edge (1);
  \path[->] (5) edge (4);
  \path[->] (6) edge (0);
  \path[->] (7) edge (6);
  \path[->] (8) edge (6);
  \path[->] (9) edge (8);
  \path[->] (10) edge (6);
  \path[->] (11) edge (10);
  \path[->] (12) edge (11);
}
\end{adjustbox}

    \column{0.5\textwidth}
{
  \small
  \color{gray}
Data downloaded from United States Geological Survey (USGS) National Water Information System (NWIS) and aggregated. For Scioto River at Columbus, Ohio.
}
  \end{columns}
\end{frame}

\begin{frame}{Data Visualized}

  Looking at the data, we can see the values vary between years.

  \begin{columns}
    \column{0.5\textwidth}
    \includegraphics[width=\textwidth]{./images/annual-plot.png}
    \pause
    
    \column{0.47\textwidth}
    \includegraphics[width=\textwidth]{./images/annual-scatter.png}
  \end{columns}
  
     But there is overall linear relation between area (\(x\)) and streamflow (\(y_j\)).
  
\end{frame}

\begin{frame}{Physics Behind the Process}
\begin{columns}
  \column{0.45\textwidth}
  \includegraphics[width=\textwidth]{./images/concept.pdf}

  \pause
  
  \column{0.45\textwidth}
  \includegraphics[width=\textwidth]{./images/bucket1.pdf}
  
  \pause

  \includegraphics[width=\textwidth]{./images/bucket2.pdf}
  
\end{columns}

  $\frac{y_1}{x_1} = \frac{y_2}{x_2} = \beta$
  
\end{frame}

\section{Methods}

\begin{frame}{Possible Models}

  Given streamflow at month \(j\) on node \(i\) is \(y_{i,j}\) and the area at node \(i\) is \(x_{i}\).

  \begin{itemize}
  \item Linear Model with different coefficient for each year,
    \begin{equation*}
      y_{i,j} = (\beta_{j} + \epsilon_{i,j}) \times x_{i}
    \end{equation*}
    
  \item Linear Model with correlation to Input Nodes
    \begin{equation*}
      y_{i,j} = (\beta \times i_{i,j} + \epsilon_{i,j}) \times x_{i,j}
    \end{equation*}
    
  \item Linear Model with correlation to Output Node
    \begin{equation*}
      y_{i,j} = (\beta \times o_{i,j} + \epsilon_{i,j}) \times x_{i,j}
    \end{equation*}

  \item Correlation with Both ({\color{red}Abandoned} as RJAGS can't handle circular dependency)
  \end{itemize}

\end{frame}

\begin{frame}[containsverbatim]{Hierarchical Linear Model}

  We have 20 linear coefficients (\(\beta_1,\hdots,\beta_{20}\)) for each year that depend on two parameters (\(\beta, \tau\)) and error (\(\sigma\)).

  Equation:
    \begin{equation*}
      y_{i,j} = \beta_{j} \times x_{i} + \epsilon_{i,j}
    \end{equation*}

    \vspace{0.2cm}
    
\inputminted{jags}{jags/linear.jags}
\end{frame}

\begin{frame}{Linear with Output}

  Here we have only 2 parameters: Linear coefficient \(\beta\), and error \(\sigma\).

    \begin{equation*}
      y_{i,j} = (\beta \times o_{i,j}) \times x_{i} + \epsilon_{i,j}
    \end{equation*}

    Here, \(o_{i,j} = y_{k,j} / x_k\) where node \(k\) is the output of node \(i\)
\end{frame}

\begin{frame}[containsverbatim]{Linear with Output}

  \inputminted{jags}{jags/output.jags}
\end{frame}


\begin{frame}[containsverbatim]{Linear with Inputs}
  \inputminted{jags}{jags/inputs.jags}
\end{frame}

\section{Results}

\begin{frame}{Single Linear Model}

  This is basically: $y_{i,j} = \beta \times x_{i} + \epsilon_{i,j}$

  That is, no hierchical model all years have same \(\beta\).
  \pause{}
  
  \begin{columns}
    \column{0.45\textwidth}
    \includegraphics[width=\textwidth]{./images/basic-beta.pdf}
    
    \column{0.45\textwidth}
    \includegraphics[width=\textwidth]{./images/basic-sigma.pdf}
  \end{columns}

  This acts as a benchmark for our other models.
\end{frame}

\begin{frame}{Linear}
  \begin{columns}
    \column{0.6\textwidth}
    \includegraphics[width=\textwidth]{./images/linear-beta.pdf}
    \pause
    
    \column{0.3\textwidth}
    \includegraphics[width=\textwidth]{./images/linear-sigma.pdf}
  \end{columns}
\end{frame}


\begin{frame}{Output and Inputs}
  \begin{columns}
    \column{0.45\textwidth} \includegraphics[width=\textwidth]{./images/output-beta.pdf}    
    \column{0.45\textwidth}
    \includegraphics[width=\textwidth]{./images/output-sigma.pdf}
  \end{columns}
  
    \pause
  \begin{columns}
    \column{0.45\textwidth}
    \includegraphics[width=\textwidth]{./images/inputs-beta.pdf}
    
    \column{0.45\textwidth}
    \includegraphics[width=\textwidth]{./images/inputs-sigma.pdf}
  \end{columns}
  
\end{frame}

\begin{frame}{Example Data (Node 6)}

  \begin{columns}
    \column{0.5\textwidth}
    Linear\\
    \includegraphics[width=0.9\textwidth]{./images/linear-node6.pdf}
    \pause

    Output\\
    \includegraphics[width=0.9\textwidth]{./images/output-node6.pdf}

    \pause
    \column{0.5\textwidth}

    Input\\
    \includegraphics[width=0.9\textwidth]{./images/inputs-node6.pdf}

    \pause
    Combined\\
    \includegraphics[width=1.05\textwidth,height=1.05in]{./images/combined-node6.pdf}
  \end{columns} 
\end{frame}


\section{Missing Data}
\begin{frame}[containsverbatim]{Simulated Missing Data}
  \begin{columns}

    \column{0.65\textwidth}
\begin{minted}{r}
valid.nodes <- c(1, 2, 4, 6, 8, 10, 11) + 1
valid.nodes.count <- length(valid.nodes)
gaps.each <- 2
## node, year, value
gaps <- matrix(
         nrow=valid.nodes.count * gaps.each,
         ncol=3
        )
ind <- 1
y <- y.org
for (i in valid.nodes) {
  for (j in sample(1:m, gaps.each)) {
    gaps[ind, ] <- c(i, j, y.org[i,j])
    y[i,j] <- NA
    ind <- ind+1
  }
}
\end{minted}

    \column{0.34\textwidth}
    
    The methods cannot be applied to all the nodes.

    \vspace{0.4cm}
    
    \includegraphics[width=\textwidth]{./graphics/gaps.pdf}
  \end{columns}
\end{frame}

\begin{frame}{Missing Data Inference}

  \begin{columns}
    \column{0.3\textwidth}
      Linear
    \includegraphics[width=\textwidth]{./images/linear-missing.pdf}
    
    \column{0.3\textwidth}
      Output
    \includegraphics[width=\textwidth]{./images/output-missing.pdf}
    \column{0.3\textwidth}
      Inputs
    \includegraphics[width=\textwidth]{./images/inputs-missing.pdf}
  \end{columns} 
\end{frame}

\section{Discussions}

\begin{frame}{Discussions}

  Results:
  \begin{itemize}
  \item All methods seem to work well to represent the data.
  \item Network based methods do not need a linear coefficient which could be a computational advantage.
  \item Missing data inference were equivalent.
  \end{itemize} 

  \pause
  Future Recommendations:
  \begin{itemize}
  \item Look into how much data can be missing to infer the rest of the data,
  \item Try different distributions for residues,
  \item Find a way to run inputs and outputs combined,
  \item Try it with autocorrelated data instead of iid, etc.
  \end{itemize}

\end{frame}

\end{document}
